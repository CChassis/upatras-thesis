% -*-latex-*-
% 
% For questions, comments, concerns or complaints:
% thesis@mit.edu
% 
%
% $Log: cover.tex,v $
% Revision 1.9  2019/08/06 14:18:15  cmalin
% Replaced sample content with non-specific text.
%
% Revision 1.8  2008/05/13 15:02:15  jdreed
% Degree month is June, not May.  Added note about prevdegrees.
% Arthur Smith's title updated
%
% Revision 1.7  2001/02/08 18:53:16  boojum
% changed some \newpages to \cleardoublepages
%
% Revision 1.6  1999/10/21 14:49:31  boojum
% changed comment referring to documentstyle
%
% Revision 1.5  1999/10/21 14:39:04  boojum
% *** empty log message ***
%
% Revision 1.4  1997/04/18  17:54:10  othomas
% added page numbers on abstract and cover, and made 1 abstract
% page the default rather than 2.  (anne hunter tells me this
% is the new institute standard.)
%
% Revision 1.4  1997/04/18  17:54:10  othomas
% added page numbers on abstract and cover, and made 1 abstract
% page the default rather than 2.  (anne hunter tells me this
% is the new institute standard.)
%
% Revision 1.3  93/05/17  17:06:29  starflt
% Added acknowledgements section (suggested by tompalka)
% 
% Revision 1.2  92/04/22  13:13:13  epeisach
% Fixes for 1991 course 6 requirements
% Phrase "and to grant others the right to do so" has been added to 
% permission clause
% Second copy of abstract is not counted as separate pages so numbering works
% out
% 
% Revision 1.1  92/04/22  13:08:20  epeisach

% NOTE:
% These templates make an effort to conform to the MIT Thesis specifications,
% however the specifications can change. We recommend that you verify the
% layout of your title page with your thesis advisor and/or the MIT 
% Libraries before printing your final copy.
\title{Development and Implementation of an Unmanned Aerial System for a 5G Vertical Use Case}

\author{Christos Chasis}
% If you wish to list your previous degrees on the cover page, use the 
% previous degrees command:
%       \prevdegrees{A.A., Harvard University (1985)}
% You can use the \\ command to list multiple previous degrees
%       \prevdegrees{B.S., University of California (1978) \\
%                    S.M., Massachusetts Institute of Technology (1981)}
\department{Department of Electrical and Computer Engineering}

% If the thesis is for two degrees simultaneously, list them both
% separated by \and like this:
% \degree{Doctor of Philosophy \and Master of Science}
\degree{Integrated Masters in Electrical and Computer Engineering}

% As of the 2007-08 academic year, valid degree months are September, 
% February, or June.  The default is June.
\degreemonth{March}
\degreeyear{2021}
\thesisdate{March 10, 2021}

%% By default, the thesis will be copyrighted to MIT.  If you need to copyright
%% the thesis to yourself, just specify the `vi' documentclass option.  If for
%% some reason you want to exactly specify the copyright notice text, you can
%% use the \copyrightnoticetext command.  
%\copyrightnoticetext{\copyright IBM, 1990.  Do not open till Xmas.}

% If there is more than one supervisor, use the \supervisor command
% once for each.
\supervisor{Spyros Denazis}{Professor}
\supervisor{Grigorios Kalivas}{Professor}

% This is the department committee chairman, not the thesis committee
% chairman.  You should replace this with your Department's Committee
% Chairman.
\chairman{Kyriakos Sgarbas}{Director, Division of Telecommunications \\ and Information Technology}

% Make the titlepage based on the above information.  If you need
% something special and can't use the standard form, you can specify
% the exact text of the titlepage yourself.  Put it in a titlepage
% environment and leave blank lines where you want vertical space.
% The spaces will be adjusted to fill the entire page.  The dotted
% lines for the signatures are made with the \signature command.
\maketitle

% The abstractpage environment sets up everything on the page except
% the text itself.  The title and other header material are put at the
% top of the page, and the supervisors are listed at the bottom.  A
% new page is begun both before and after.  Of course, an abstract may
% be more than one page itself.  If you need more control over the
% format of the page, you can use the abstract environment, which puts
% the word "Abstract" at the beginning and single spaces its text.

%% You can either \input (*not* \include) your abstract file, or you can put
%% the text of the abstract directly between the \begin{abstractpage} and
%% \end{abstractpage} commands.

% First copy: start a new page, and save the page number.
\cleardoublepage
% Uncomment the next line if you do NOT want a page number on your
% abstract and acknowledgments pages.
% \pagestyle{empty}
\setcounter{savepage}{\thepage}
\begin{abstractpage}
% $Log: abstract.tex,v $
% Revision 1.1  93/05/14  14:56:25  starflt
% Initial revision
% 
% Revision 1.1  90/05/04  10:41:01  lwvanels
% Initial revision
% 
%
%% The text of your abstract and nothing else (other than comments) goes here.
%% It will be single-spaced and the rest of the text that is supposed to go on
%% the abstract page will be generated by the abstractpage environment.  This
%% file should be \input (not \include 'd) from cover.tex.
Recent advances in the utilization of semi or fully autonomous platforms have created an emerging area of interest regarding their utilization in various aspects of  applications. Despite that fact, network connectivity, a crucial part to the successful application of their benefits, is still in early research stages. 
In this work, an overview of current trends and technologies regarding cellular networks is presented, followed by a detailed description of a system architecture that aims to showcase the feasibility of a remote-controlled vehicle over a network. Emphasis has been given to engineering a complete end-to-end system that is comprised of primarily open-source components, while being forwards compatible in regards to experimental or commercial 5G deployments.
\end{abstractpage}

% Additional copy: start a new page, and reset the page number.  This way,
% the second copy of the abstract is not counted as separate pages.
% Uncomment the next 6 lines if you need two copies of the abstract
% page.
% \setcounter{page}{\thesavepage}
% \begin{abstractpage}
% % $Log: abstract.tex,v $
% Revision 1.1  93/05/14  14:56:25  starflt
% Initial revision
% 
% Revision 1.1  90/05/04  10:41:01  lwvanels
% Initial revision
% 
%
%% The text of your abstract and nothing else (other than comments) goes here.
%% It will be single-spaced and the rest of the text that is supposed to go on
%% the abstract page will be generated by the abstractpage environment.  This
%% file should be \input (not \include 'd) from cover.tex.
Recent advances in the utilization of semi or fully autonomous platforms have created an emerging area of interest regarding their utilization in various aspects of  applications. Despite that fact, network connectivity, a crucial part to the successful application of their benefits, is still in early research stages. 
In this work, an overview of current trends and technologies regarding cellular networks is presented, followed by a detailed description of a system architecture that aims to showcase the feasibility of a remote-controlled vehicle over a network. Emphasis has been given to engineering a complete end-to-end system that is comprised of primarily open-source components, while being forwards compatible in regards to experimental or commercial 5G deployments.
% \end{abstractpage}

\cleardoublepage

\section*{Acknowledgments}

I would like to wholeheartedly thank my thesis supervisor Spyros Denazis for entrusting me with the responsibility of developing part of the 5G-VINNI use case deliverable and providing me with invaluable resources and guidance to refine and materialize my task. Without his contributions, I would not have the privilege of working with a distinguished research team such as the Network Architectures and Management group, nor learn as much in terms of leadership and project management. \\Additionally, I would like to thank my thesis co-supervisor Grigorios Kalivas, which instilled me the principles of a proper engineer and a passion for wireless technologies, way before I even was sure about what I wanted to pursue for my professional career. It has been truly a pleasure cooperating with him, from past design contests up to the biggest task so far of my academic endeavors. 
\\Last, but certainly not least, I should credit the people I had the privilege of working alongside, which brought (and still bring) in their own ways their countless years of experience towards making the first experimental 5G facility in Greece a reality. These are in no particular order Christos Tranoris, Panagiotis Papaioannou, Takis Apostolopoulos and Athanasios Chamalidis.

\begin{center}
    \thispagestyle{empty}
    \vspace*{\fill}
    This work is dedicated to those who seek to push the boundaries of technology and through their hard work, help make our everyday lives a little bit better.
    \vspace*{\fill}
\end{center}


%%%%%%%%%%%%%%%%%%%%%%%%%%%%%%%%%%%%%%%%%%%%%%%%%%%%%%%%%%%%%%%%%%%%%%
% -*-latex-*-
